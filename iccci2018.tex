\documentclass{llncs}

\usepackage{graphicx}
\usepackage[hyphens]{url}
\usepackage{booktabs}
\usepackage{paralist}
\usepackage{multirow}

% natbib for refs
%\usepackage[numbers,sort]{natbib} 

\begin{document}

\title{Investigation the relationship between personality traits, six basic emotions and gender with respect to Age}

\author{Mohamed Mostafa\inst{1} \and Tom
  Crick\inst{2} \and Ana C. Calderon\inst{2} \and Giles Oatley\inst{3}}


\institute{Cardiff University, UK\\\email{mostafam1@cardiff.ac.uk}
\and 
Cardiff Metropolitan University,
UK\\\email{\{tcrick,acalderon\}@cardiffmet.ac.uk}
\and
Murdoch University, Australia\\\email{g.oatley@murdoch.edu.au}}

\maketitle

\begin{abstract}
Abstract here...
 \end{abstract}

\begin{keywords}
TBC
\end{keywords}

\section{Introduction}\label{intro}

The previous experiments suggested a strong correlation between personality traits and emotions, furthermore, the attempt of modelling server status, suggested a strong and potenial method to model the user behaviour in different complex system behaviour. This analysis to explore the big personality traits and emotions assosication and correlation, further correlation between Gender/Age and Personality traits - Emotions. According to \cite{10.1371/journal.pone.0073791} Gender and Age correlate with the personality traits however, in same study the emotion features was not included. Therefore, this experiment is essential in cross-validating the methodology used in the \cite{10.1371/journal.pone.0073791} and add more features to the model to check if emotions can play a positive role in the equation.

\subsection{Data set}
The same data set will be used in the analysis as Personality Traits retrieved from the Motivation letter and emotions retrieved from different platform that was used in communication (i.e: Facebook, HelpDesk). The basic information will be retrieved to include the gender and age as extra parameters and the analysis will be running separately.


\subsection{Binomial Logistic Regression}\label{BioGenderAge}

The data set combination suggested to use Binomial Logistic Regression, as the dataset similar to ~\ref{subs:sentimentStages}, with Gender instead of Stages ID. The experiment is to investigate the probability of being able to predicate the gender based on Big Five traits and Emotions, and according to the result, it will be decided whether to include the Gender as controlling variable in the conceptual model. Adding the Age variable alongside with Big Five Traits and Emotions to investigate if it would improve the model or not.

In order to apply, Binomial Logistic Regression, the data needs to pass the following assumptions:

\begin{itemize}
\item
Linear relationship between the Big Five Traits, Emotions and logit transformation of the gender variable. 
\item
Data must not show multicollinearity
\item
There should be no significant outliers, high leverage points or highly influential points
\end{itemize}

\paragraph{Linear relationship between the Big Five Traits, Emotions and logit transformation of the \textbf{gender} variable}

The first part of the Box-Tidwell (1962) procedure requires that all continuous independent variables are first transformed into their natural logs, this means that we need to perform natural log transformations on our continuous independent variables: Big Five Traits and Emotions.The second part of the Box-Tidwell (1962) procedure requires that you create interaction terms for each of your continuous independent variables and their respective natural log transformed variables. Since we have three continuous independent variables in our example, this means that we have to create Big Five Trait and Emotions - interaction terms: \verb|ln_sadness| \mbox{*} sadness (i.e., the product of \verb|ln_sadness| by sadness – then need to be entered into the binomial logistic regression procedure, together with the gender and age.

According to \cite{Tabachnick2001}, to calculate the new alpha (α) level (i.e., p-value) for current dataset , it is by dividing the alpha level (p < .05) by the number of terms in your model. Formulaically, this is:

\begin{equation}
adjusted alpha level = \frac{OriginalAlphaLevel}{numberofComparisons}
\end{equation}

The new adjusted alpha level in this case is \textit{0.002}, (i.e., 0.05 / 23= 0.002). Linearity of the Big Five Traits and Emotions with respect to the logit of the \textit{Gender} variable was assessed via the Box-Tidwell (1962) procedure. A Bonferroni correction was applied using all twenty-one terms in the model resulting in statistical significance being accepted when p \mbox{<} .002 \cite{Tabachnick2001}. According to the table ~\ref{tbl:LineraityCheckGender} , all continuous independent variables were found to be linearly related to the logit of the dependent variable.


% Please add the following required packages to your document preamble:
% \usepackage{booktabs}
\begin{table}[!ht]
\centering
\begin{tabular}{@{}lllllll@{}}
\toprule
\multicolumn{7}{c}{\textbf{Variables in the Equation}}                                          \\ \midrule
                                           & B      & S.E.   & Wald  & df    & Sig.  & Exp(B)   \\
Step 1a                                    & Anger  & 1.248  & 2.244 & 0.309 & 1     & 0.578    \\
Disgust                                    & -0.332 & 23.43  & 0     & 1     & 0.989 & 0.717    \\
Fear                                       & 1.667  & 2.822  & 0.349 & 1     & 0.555 & 5.294    \\
Joy                                        & 1.961  & 1.496  & 1.719 & 1     & 0.19  & 7.107    \\
Sadness                                    & -1.251 & 0.854  & 2.145 & 1     & 0.143 & 0.286    \\
Openness                                   & -2.494 & 1.152  & 4.691 & 1     & 0.03  & 0.083    \\
Conscientiousness                          & 2.303  & 1.266  & 3.311 & 1     & 0.069 & 10.008   \\
Extraversion                               & 0.484  & 0.911  & 0.282 & 1     & 0.595 & 1.622    \\
Agreeableness                              & 0.155  & 0.954  & 0.026 & 1     & 0.871 & 1.167    \\
Neuroticism                                & 0.166  & 0.832  & 0.04  & 1     & 0.842 & 1.181    \\
Age                                        & 0.289  & 0.584  & 0.245 & 1     & 0.621 & 1.335    \\
Anger by ln\_anger                         & 1.945  & 3.032  & 0.412 & 1     & 0.521 & 6.992    \\
Disgust by ln\_disgust                     & 2.757  & 11.198 & 0.061 & 1     & 0.806 & 15.75    \\
Fear by ln\_fear                           & -0.623 & 3.262  & 0.036 & 1     & 0.849 & 0.536    \\
Joy by ln\_joy                             & 4.399  & 2.49   & 3.121 & 1     & 0.077 & 81.354   \\
Sadness by ln\_sadness                     & -1.914 & 2.128  & 0.81  & 1     & 0.368 & 0.147    \\
Openness by ln\_openness                   & 7.291  & 3.366  & 4.693 & 1     & 0.03  & 1466.912 \\
Conscientiousness by ln\_conscientiousness & 1.096  & 2.669  & 0.169 & 1     & 0.681 & 2.992    \\
Extraversion by ln\_extraversion           & -3.599 & 2.282  & 2.487 & 1     & 0.115 & 0.027    \\
Agreeableness by ln\_agreeableness         & -0.662 & 2.413  & 0.075 & 1     & 0.784 & 0.516    \\
Neuroticism by ln\_neuroticism             & 1.957  & 2.622  & 0.557 & 1     & 0.456 & 7.076    \\
Age by ln\_age                             & -0.053 & 0.129  & 0.17  & 1     & 0.68  & 0.948    \\
Constant                                   & -0.39  & 4.891  & 0.006 & 1     & 0.936 & 0.677    \\ \bottomrule
\end{tabular}
\caption{Variables in the Equation -  Gender}
\label{tbl:LineraityCheckGender}
\end{table}

\paragraph{Data must not show multicollinearity}, next step to investigate if the data shows or does not show multicollinearity to validate the possibility of applying binomial logistic regression. According to table ~\ref{tbl:GenderCasewiseDiagnostics} there was one studentized residual with a value of -2.376743 standard deviations, which was kept in the analysis.

% Please add the following required packages to your document preamble:
% \usepackage{booktabs}
\begin{table}[!ht]
\centering
\begin{tabular}{@{}lllllll@{}}
\toprule
\multicolumn{7}{c}{\textbf{Casewise Listb}}                                                      \\ \midrule
Case   & Selected Statusa & Observed & Predicted & Predicted Group & Temporary Variable &        \\
gender &                  &          &           &                 & Resid              & ZResid \\
7      & S                & F**      & 0.85      & M               & -0.85              & -2.377 \\ \bottomrule
\end{tabular}
\caption{Casewise Diagnostics}
\label{tbl:GenderCasewiseDiagnostics}
\end{table}


\subsubsection{Bionomial Findings}
This aim of this experiment is to investigate which variable were statistically significant of the Big Five Traits, Emotions and Age with respect to the Gender, only three were statistically significant: Openness (p<0.072), Conscientiousness and Age (as shown in Table ~\ref{tbl:VariablesintheEquation}). The result reported does not give enough accuracy regarding the correlation between Big Five Traits, Emotions and Age to predict the Gender. Therefore, another form of analysis is applied next to explore and investigate potential association between the above variables.

% \usepackage{booktabs}
% \usepackage{multirow}
\begin{table}[!ht]
\centering
\begin{tabular}{@{}lllllllll@{}}
\toprule
\multicolumn{9}{c}{\textbf{Variables in the Equation}}                                                                                                                                          \\ \midrule
                  & \multirow{2}{*}{B} & \multirow{2}{*}{S.E.} & \multirow{2}{*}{Wald} & \multirow{2}{*}{df} & \multirow{2}{*}{Sig.} & \multirow{2}{*}{Exp(B)} & 95\% C.I.for EXP(B) &          \\
                  &                    &                       &                       &                     &                       &                         & Lower               & Upper    \\
Anger             & 0.329              & 1.518                 & 0.047                 & 1                   & 0.828                 & 1.39                    & 0.071               & 27.209   \\
Disgust           & -5.201             & 7.187                 & 0.524                 & 1                   & 0.469                 & 0.006                   & 0                   & 7219.283 \\
Fear              & 1.405              & 1.635                 & 0.739                 & 1                   & 0.39                  & 4.075                   & 0.165               & 100.404  \\
Joy               & 0.169              & 0.951                 & 0.031                 & 1                   & 0.859                 & 1.184                   & 0.184               & 7.631    \\
Sadness           & -0.567             & 0.731                 & 0.6                   & 1                   & 0.438                 & 0.567                   & 0.135               & 2.379    \\
Openness          & -1.556             & 0.866                 & 3.229                 & 1                   & 0.072                 & 0.211                   & 0.039               & 1.152    \\
Conscientiousness & 1.261              & 0.764                 & 2.722                 & 1                   & 0.099                 & 3.529                   & 0.789               & 15.786   \\
Extraversion      & 0.389              & 0.78                  & 0.249                 & 1                   & 0.618                 & 1.476                   & 0.32                & 6.801    \\
Agreeableness     & 0.262              & 0.778                 & 0.113                 & 1                   & 0.736                 & 1.3                     & 0.283               & 5.974    \\
Neuroticism       & 0.234              & 0.644                 & 0.132                 & 1                   & 0.717                 & 1.263                   & 0.358               & 4.46     \\
Age               & 0.059              & 0.031                 & 3.664                 & 1                   & 0.056                 & 1.061                   & 0.999               & 1.127    \\
Constant          & -1.395             & 1.175                 & 1.41                  & 1                   & 0.235                 & 0.248                   &                     &          \\ \bottomrule
\end{tabular}
\caption{Binomial Log - Variables in the Equation}
\label{tbl:VariablesintheEquation}
\end{table}



\subsection{Pearson's partial correlation}
As the Binomial Logistic Regression, suggested a correlation between \emph{Openness}, \emph{Conscientiousness} and \emph{Age} to predict the \emph{Gender}, the Pearson's partial correlation was run to assess the relationship between Big Five Traits, Emotions, Age and Gender and to confirm the output of the Binomial or include more variable as strong association.

According to analysis performed in ~\ref{subs:ProfilingComplex}, there were linear relationships between Big Five Traits and Emotions, as assessed by scatterplots and partial regression plots. There was univariate normality, as assessed by Shapiro-Wilk's test (p > .05), and there were no univariate or multivariate outliers, as assessed by Mahalanobis Distance respectively.~\ref{fig:normalpp}~\ref{fig:scatterplot} 

% Please add the following required packages to your document preamble:
% \usepackage{booktabs}
\begin{table}[]
\centering
\begin{tabular}{@{}lllllll@{}}
\toprule
\multicolumn{7}{c}{Pearson's partial correlation}                                       \\ \midrule
                  &                         & Anger & Disgust & Fear  & Joy   & Sadness \\
\multicolumn{7}{c}{\textbf{Controlling Variable: None}}                                 \\
Openness          & Correlation             & .044  & -.008   & -.038 & .024  & -.015   \\
                  & Significance (2-tailed) & .529  & .913    & .586  & .729  & .833    \\
                  & df                      & 204   & 204     & 204   & 204   & 204     \\
Conscientiousness & Correlation             & -.141 & -.093   & -.099 & .057  & -.040   \\
                  & Significance (2-tailed) & .043  & .183    & .155  & .418  & .566    \\
                  & df                      & 204   & 204     & 204   & 204   & 204     \\
Extraversion      & Correlation             & .040  & .056    & -.035 & -.079 & .003    \\
                  & Significance (2-tailed) & .567  & .421    & .616  & .262  & .960    \\
                  & df                      & 204   & 204     & 204   & 204   & 204     \\
Agreeableness     & Correlation             & .010  & .041    & .026  & -.094 & .053    \\
                  & Significance (2-tailed) & .881  & .556    & .715  & .178  & .447    \\
                  & df                      & 204   & 204     & 204   & 204   & 204     \\
Neuroticism       & Correlation             & -.038 & -.058   & -.166 & -.006 & .030    \\
                  & Significance (2-tailed) & .585  & .407    & .017  & .937  & .664    \\
                  & df                      & 204   & 204     & 204   & 204   & 204     \\
Age               & Correlation             & -.042 & -.082   & -.187 & .170  & -.085   \\
                  & Significance (2-tailed) & .551  & .241    & .007  & .015  & .222    \\
                  & df                      & 204   & 204     & 204   & 204   & 204     \\
\multicolumn{7}{c}{\textbf{Controlling Variable: Gender}}                               \\
Openness          & Correlation             & .041  & -.014   & -.039 & .029  & -.020   \\
                  & Significance (2-tailed) & .558  & .844    & .579  & .681  & .772    \\
                  & df                      & 203   & 203     & 203   & 203   & 203     \\
Conscientiousness & Correlation             & -.139 & -.086   & -.100 & .052  & -.034   \\
                  & Significance (2-tailed) & .047  & .219    & .155  & .463  & .632    \\
                  & df                      & 203   & 203     & 203   & 203   & 203     \\
Extraversion      & Correlation             & .041  & .058    & -.035 & -.080 & .005    \\
                  & Significance (2-tailed) & .560  & .410    & .617  & .256  & .946    \\
                  & df                      & 203   & 203     & 203   & 203   & 203     \\
Agreeableness     & Correlation             & .013  & .046    & .026  & -.098 & .057    \\
                  & Significance (2-tailed) & .855  & .514    & .711  & .163  & .413    \\
                  & df                      & 203   & 203     & 203   & 203   & 203     \\
Neuroticism       & Correlation             & -.036 & -.054   & -.166 & -.008 & .034    \\
                  & Significance (2-tailed) & .606  & .439    & .017  & .904  & .627    \\
                  & df                      & 203   & 203     & 203   & 203   & 203     \\
Age               & Correlation             & -.038 & -.075   & -.188 & .166  & -.080   \\
                  & Significance (2-tailed) & .587  & .282    & .007  & .017  & .257    \\
                  & df                      & 203   & 203     & 203   & 203   & 203     \\
                  &                         &       &         &       &       &         \\ \bottomrule
\end{tabular}
\caption{Pearson's Partial correlation - Controlling Variable - Gender }
\label{tbl: PearControllingGender}
\end{table}

\begin{table}[!ht]
\centering
\begin{tabular}{lllllll}
\hline
\multicolumn{7}{c}{Pearson's partial correlation}                                       \\ \hline
                  &                         & Anger & Disgust & Fear  & Joy   & Sadness \\
\multicolumn{7}{c}{Controlling Variable: None}                                          \\
Openness          & Correlation             & .044  & -.008   & -.038 & .024  & -.015   \\
                  & Significance (2-tailed) & .529  & .913    & .586  & .729  & .833    \\
                  & df                      & 204   & 204     & 204   & 204   & 204     \\
Conscientiousness & Correlation             & -.141 & -.093   & -.099 & .057  & -.040   \\
                  & Significance (2-tailed) & .043  & .183    & .155  & .418  & .566    \\
                  & df                      & 204   & 204     & 204   & 204   & 204     \\
Extraversion      & Correlation             & .040  & .056    & -.035 & -.079 & .003    \\
                  & Significance (2-tailed) & .567  & .421    & .616  & .262  & .960    \\
                  & df                      & 204   & 204     & 204   & 204   & 204     \\
Agreeableness     & Correlation             & .010  & .041    & .026  & -.094 & .053    \\
                  & Significance (2-tailed) & .881  & .556    & .715  & .178  & .447    \\
                  & df                      & 204   & 204     & 204   & 204   & 204     \\
Neuroticism       & Correlation             & -.038 & -.058   & -.166 & -.006 & .030    \\
                  & Significance (2-tailed) & .585  & .407    & .017  & .937  & .664    \\
                  & df                      & 204   & 204     & 204   & 204   & 204     \\
Gender            & Correlation             & .030  & .056    & .005  & -.041 & .050    \\
                  & Significance (2-tailed) & .668  & .423    & .947  & .558  & .473    \\
                  & df                      & 204   & 204     & 204   & 204   & 204     \\
Age               & Correlation             & -.042 & -.082   & -.187 & .170  & -.085   \\
                  & Significance (2-tailed) & .551  & .241    & .007  & .015  & .222    \\
                  & df                      & 204   & 204     & 204   & 204   & 204     \\
\multicolumn{7}{c}{Controlling Variable Age}                                            \\
Openness          & Correlation             & .050  & .003    & -.014 & .002  & -.004   \\
                  & Significance (2-tailed) & .475  & .964    & .842  & .978  & .960    \\
                  & df                      & 203   & 203     & 203   & 203   & 203     \\
Conscientiousness & Correlation             & -.138 & -.084   & -.080 & .038  & -.031   \\
                  & Significance (2-tailed) & .049  & .229    & .257  & .592  & .664    \\
                  & df                      & 203   & 203     & 203   & 203   & 203     \\
Extraversion      & Correlation             & .041  & .058    & -.032 & -.083 & .005    \\
                  & Significance (2-tailed) & .559  & .407    & .650  & .235  & .940    \\
                  & df                      & 203   & 203     & 203   & 203   & 203     \\
Agreeableness     & Correlation             & .001  & .023    & -.018 & -.057 & .035    \\
                  & Significance (2-tailed) & .990  & .743    & .796  & .413  & .622    \\
                  & df                      & 203   & 203     & 203   & 203   & 203     \\
Neuroticism       & Correlation             & -.034 & -.051   & -.152 & -.022 & .039    \\
                  & Significance (2-tailed) & .623  & .471    & .030  & .753  & .580    \\
                  & df                      & 203   & 203     & 203   & 203   & 203     \\
Gender            & Correlation             & .025  & .046    & -.020 & -.019 & .040    \\
                  & Significance (2-tailed) & .724  & .513    & .772  & .785  & .574    \\
                  & df                      & 203   & 203     & 203   & 203   & 203     \\
\multicolumn{7}{c}{\textbf{Cells contain zero-order (Pearson) correlations.}}           \\ \hline
\end{tabular}
\caption{Pearson's Partial correlation - Controlling Variable - Age }
\label{tbl: PearControllingGender}
\end{table}

% Please add the following required packages to your document preamble:
% \usepackage{booktabs}
\begin{table}[!ht]
\centering
\begin{tabular}{@{}lllllll@{}}
\toprule
\multicolumn{7}{c}{\textbf{Pearson's partial correlation}}                              \\ \midrule
Correlation       &                         & Anger & Disgust & Fear  & Joy   & Sadness \\
\multicolumn{7}{c}{\textbf{Controlling Variable: None}}                                 \\
Openness          & Correlation             & .044  & -.008   & -.038 & .024  & -.015   \\
                  & Significance (2-tailed) & .529  & .913    & .586  & .729  & .833    \\
                  & df                      & 204   & 204     & 204   & 204   & 204     \\
Conscientiousness & Correlation             & -.141 & -.093   & -.099 & .057  & -.040   \\
                  & Significance (2-tailed) & .043  & .183    & .155  & .418  & .566    \\
                  & df                      & 204   & 204     & 204   & 204   & 204     \\
Extraversion      & Correlation             & .040  & .056    & -.035 & -.079 & .003    \\
                  & Significance (2-tailed) & .567  & .421    & .616  & .262  & .960    \\
                  & df                      & 204   & 204     & 204   & 204   & 204     \\
Agreeableness     & Correlation             & .010  & .041    & .026  & -.094 & .053    \\
                  & Significance (2-tailed) & .881  & .556    & .715  & .178  & .447    \\
                  & df                      & 204   & 204     & 204   & 204   & 204     \\
Neuroticism       & Correlation             & -.038 & -.058   & -.166 & -.006 & .030    \\
                  & Significance (2-tailed) & .585  & .407    & .017  & .937  & .664    \\
                  & df                      & 204   & 204     & 204   & 204   & 204     \\
Age               & Correlation             & -.042 & -.082   & -.187 & .170  & -.085   \\
                  & Significance (2-tailed) & .551  & .241    & .007  & .015  & .222    \\
                  & df                      & 204   & 204     & 204   & 204   & 204     \\
Gender       & Correlation             & .030  & .056    & .005  & -.041 & .050    \\
                  & Significance (2-tailed) & .668  & .423    & .947  & .558  & .473    \\
                  & df                      & 204   & 204     & 204   & 204   & 204     \\
\multicolumn{7}{c}{\textbf{Controlling Variable: Age and Gender}}                       \\
Openness          & Correlation             & .047  & -.003   & -.012 & .004  & -.009   \\
                  & Significance (2-tailed) & .501  & .969    & .870  & .950  & .902    \\
                  & df                      & 202   & 202     & 202   & 202   & 202     \\
Conscientiousness & Correlation             & -.136 & -.079   & -.083 & .036  & -.026   \\
                  & Significance (2-tailed) & .053  & .260    & .239  & .614  & .714    \\
                  & df                      & 202   & 202     & 202   & 202   & 202     \\
Extraversion      & Correlation             & .042  & .059    & -.032 & -.084 & .006    \\
                  & Significance (2-tailed) & .554  & .399    & .646  & .233  & .930    \\
                  & df                      & 202   & 202     & 202   & 202   & 202     \\
Agreeableness     & Correlation             & .004  & .028    & -.021 & -.060 & .039    \\
                  & Significance (2-tailed) & .958  & .686    & .770  & .394  & .576    \\
                  & df                      & 202   & 202     & 202   & 202   & 202     \\
Neuroticism       & Correlation             & -.033 & -.048   & -.153 & -.023 & .041    \\
                  & Significance (2-tailed) & .639  & .495    & .029  & .741  & .557    \\
                  & df                      & 202   & 202     & 202   & 202   & 202     \\
\multicolumn{7}{l}{a Cells contain zero-order (Pearson) correlations.}                  \\ \bottomrule
\end{tabular}
\caption{Pearson's Partial correlation - Controlling Variable - Gender and Age }
\label{tbl: PearControllingGenderAge}
\end{table}


\subsubsection{Findings and Discussion}

The above tables shows the output of Pearson's Partial Correlation. In table ~\ref{tbl: PearControllingGender}, the controlling variable is \emph{Gender}, a bivariate Pearson's correlation established that there was a strong, statistically significant linear relationship between \emph{Conscientiousness} and \emph{Anger}, r(204) = -.141, p < .05 , \emph{Neuroticism} and \emph{Fear} r(204) = -.166, p < .05. Pearson's partial correlation showed that the strength of this linear relationship was improved when \emph{Gender} was controlled for  \emph{Conscientiousness} and \emph{Anger} rpartial(203) = -.139, p=0.47 and it is still the same between \emph{Neuroticism} and \emph{Fear}  rpartial(203) = -.166 - p=.017 and still statistically significant. In table ~\ref{tbl: PearControllingAge}, the controlling variable is \emph{Age}. Pearson's partial correlation showed that the strength of this linear relationship was improved when \emph{Age} was controlled, in respect to the relationship between \emph{Conscientiousness} and \emph{Anger}  rpartial(203) = -.138 - p=0.49 and between \emph{Neuroticism} and \emph{Fear} rpartial(203) = -.152 - p=0.030  and still statistically significant. In table ~\ref{tbl: PearControllingGenderAge}, the controlling variable is \emph{Gender} and \emph{Age}. Pearson's partial correlation showed that the strength of this linear relationship was improved when \emph{Age} was controlled, in respect to the relationship between \emph{Conscientiousness} and \emph{Anger} and \emph{Neuroticism} and \emph{Fear} , rpartial(203) = -.138 - p=0.49 , and between and \emph{Neuroticism} and \emph{Fear} rpartial(203) = -.152 - p=0.030  and still statistically significant. The above findings suggests that \emph{Gender} and \emph{Age} as controlled variable combined ~\ref{tbl: PearControllingGenderAge} would improve the linear relationship between Big Five and Emotions variables specially \emph{Conscientiousness}, \emph{Neuroticism} , \emph{Anger} and \emph{Fear} and improve strength of linear relationship between \emph{Extraversion} and \emph{Anger}, \emph{Disgust}, \emph{Fear}, \emph{Joy} and \emph{Sadness} although the linear relationship was not statistically significant. Those findings are aligned with the output from the Binomial Logistic Regression \ref{BioGenderAge}, in the correlation of the \emph{Conscientiousness} and \emph{Age} and impact of \emph{Gender} in improving the association between variables.


% bib
\bibliographystyle{splncs}
\bibliography{iccci2018}

\end{document}
